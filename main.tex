\documentclass{article}

% Language setting
% Replace `english' with e.g. `spanish' to change the document language
\usepackage[english]{babel}

% Set page size and margins
% Replace `letterpaper' with`a4paper' for UK/EU standard size
\usepackage[letterpaper,top=2cm,bottom=2cm,left=3cm,right=3cm,marginparwidth=1.75cm]{geometry}

% Useful packages
\usepackage{amsmath}
\usepackage{graphicx}
\usepackage[colorlinks=true, allcolors=blue]{hyperref}

\title{A Comparitive Study Between Gaming And Gambling}
\author{Pranveer Singh}

\begin{document}
\maketitle

\begin{abstract}
This is an abstract
\end{abstract}

\section{Introduction}
Since the start of this century, we live in a world saturated with technology. Technology is foundational to modern day society. Today, an average human cannot help but to interact with some form of technology daily, that is, it is unavoidable.
This technology is also our method of connecting to the Internet. With the advent of the internet, we are able to operate in a way that would seem nothing but magical to someone over a few decades ago. Society has undeniably evolved through this, and although there are endless positives to this advancement, this report will shine light to a possibly overlooked con of this development.

The goal of this report is to compare the recent trends in \textbf{Gaming} and adjacent concepts with the established and well researched industry of \textbf{Gambling}. A gamble is defined as "an act having an element of risk" and thus gambling is formally defined as "the practice or activity of betting : the practice of risking money or other stakes in a game or bet". Traditional gambling is well known to be risky, and yet addictive. In fact, it was a highly prominent form of "gaming" in the 19th and 20th century. 

The scope of this report is far narrower than what is presumed above, yet this comparison between gaming and gambling may shine a light on the ethics of the industry and whether gaming needs restrictions similar to gambling.


\section{Conclusion}
Through the research conducted in the numerous studies cited, it is evident that there may be a closer relation between gambling and features implemented in games, and in some cases the internet as a whole, than initially expected. There is also a lack of research on the psychological impact of these attention-grabbing tactics on children, and may promote the prevalence of addictive behavior in the youth if not regulated. 


\bibliographystyle{alpha}
\bibliography{sample}

\end{document}